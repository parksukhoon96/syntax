\documentclass[12pt, a4paper]{book}
\usepackage[utf8]{inputenc}
\usepackage{fancyhdr}
\usepackage{graphicx}
\usepackage[parfill]{parskip}
\PassOptionsToPackage{hyphens}{url}\usepackage{hyperref}
\usepackage{geometry}
\usepackage{kotex}
\usepackage[
backend=biber,
style=ieee,
]{biblatex}


\pagestyle{fancy}
\fancyhf{}
\setlength{\headheight}{14pt}
\rhead{Syntax_Ch2_CPS1}
\lhead{박석훈}
\cfoot{\thepage}

\begin{document}

\title{Syntax CPS 1}
\author{박석훈}
\maketitle

\section{CPS1}
In the text we claimed that the suffixes \textit{-ian} and \textit{ish} mark adjectives. Consider the following sentences: \\

a) The Canadian government uses a parliamentary system of democracy. \\
b) The Canadian bought herself a barbeque. \\
c) The prudish linguist didn't enjoy looking at the internet. \\
d) We keep those censored copies of the book hidden to protect the sensibilities of the prudish. \\

What should we make the words ending in \textit{-ish} and \textit{-ian} in sentences (b) and (d)? Are they adjectives? If not, how can we account for the fact that these words end in \textit{-ish} and \textit{-ian}? There are many possible answers to this question



\end{document}

